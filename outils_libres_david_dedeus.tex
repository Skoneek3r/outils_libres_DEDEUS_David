\documentclass{report}

\usepackage[utf8]{inputenc}
\usepackage[0T1]{inputenc}

\usepackage[french]{babel}

\begin{document}

\chapter{Efficacité}
\section{TP 1}
 
\subsection{raccourci clavier}
\begin{tabular}{|c|c|c|}
	\hline
	\textbf{Priorité} &  \textbf{Problème} & \textbf{Correctif} \\ 
	\hline
	1 & Logout & \textsc{ctrl + alt + del} \\
	\hline
	2 & Verrouiler la session & \textsc{win + l} \\
	\hline  
	3 & Fermer un ongler & \textsc{ctrl-w} \\
	\hline
	4 & Changer d'onglet & \textsc{ctrl + tab}  \\
	\hline
	5 & Suivant ou précédent & \textsc{alt + < ou alt + >} \\
	\hline 
	6 & Ouvrir un nouvel onglet & \textsc{ctrl + t} \\
	\hline
	7 & Ouvrir une nouvelle instance & \textsc{ctrl + n} \\
	\hline 
	8 & Changer de console & \textsc{ctrl + alt + n°console} \\
	\hline 
	9 & Debut de ligne & \textsc{ctrl + a} \\
	\hline 
	10 & Fin de ligne & \textsc{ctrl + e} \\
	\hline 
	11 & Clear le terminal & \textsc{ctrl + l} \\
	\hline 
\end{tabular}


\chapter{SSH}

\section{TP 1}
\\
Pour se connecter sans utiliser \textbf{vagrant ssh}, les commandes à utilisés sont : \\
\textbf{ssh alice@10.0.0.3} \\
\textbf{ssh bob@10.0.0.3} \\
\textbf{ssh carol@10.0.0.3} \\

On est pas en local, car ce n'est pas l'addresse IP du poste local.

\section{TP 2}
\subsection{génération des clés ssh}
\textbf{debiandav@david \% ssh-keygen} \\

\subsection{copie des clés ssh}
\\ 
\textbf{debiandav@david \% sudo ssh-copy ~/.ssh/id\_rsa.pub} \\
\textbf{debiandav@david \% sudo ~/.ssh/id\_rsa bob@10.0.0.3:\./cle.txt} \\ 
\textbf{debiandav@david \% ssh bob@10.0.0.3} \\
\\
\textbf{bob@srv:~\$ mkdir .ssh} \\
\textbf{bob@srv:~\$ cat cle.txt \> ~/.ssh/authorized\_keys} \\

\subsection{définition d'une passphrase}
C'est une phrase pour s'authentifié lors d'une connexion en ssh, 
elle permet ainsi de protéger la clé publique.

\section{TP 3}
\textbf{ssh-keyscan -H 10.0.0.3 \> /home/david/.ssh/known\_hosts} \\

\subsection{fichier config}
\\
\textbf{touch .ssh/config} \\ 
\textbf{sudo chmod 600 .ssh/config} \\
\textbf{nano .ssh/config} \\
\\
\textit{Host bs} \\
\textit{Hostname 10.0.0.3} \\
\textit{Port 22} \\
\textit{User bob} \\
\\
La commande \textbf{ssh bc} fonctionne bien.

\subsection{SFTP}
\\
\textbf{debiandav@david ~ \% sftp alice@10.0.0.3} \\
\textbf{sftp\> put /home/david/outils\_libres/vagrant/README.md} \\
\textbf{sftp\> get /home/bob/cle.pub} \\
\textbf{sftp\> exit} \\

\subsection{SSHFS}
\\
\textbf{mkdir /tmp/alice\_srv} \\
\\
J'ai mit le fichier à modifier dans ce dossier. \\
\\
\textbf{debiandav@david ~ \% sshfs alice@10.0.0.3:/home/alice /tmp/alice\_srv} \\
\textbf{debiandav@david ~ \% subl /tmp/alice\_srv/README.md} \\
\\
Avec sublimetext , j'ai effacé tout le contenue en mettant "john". \\
Ensuite on vérifie en se connectant avec alice. \\
\\
\textbf{alice@srv:\~\$ cat README.md} \\
\textit{john} \\

\chapter{GIT}

\section{TP 1}
on remarque que les fichiers ne sont pas ajoutés au repository donc pour régler le problème 
on commit les modifications apportés.

\subsection{ajout des vagrant files}
\textit{git commit -m "ajout des vagrantfiles"
[master (commit racine) 0b7a33a] ajout des vagrantfiles
 3 files changed, 70 insertions(+)
 create mode 100644 vagrant/Vagrantfile
 create mode 100755 vagrant/srv/test1.cgi
 create mode 100755 vagrant/srv/test2.cgi}

\subsection{log commit}
\textit{git log commit 0b7a33a2e6f7c228dceb08d0eb079476ee5a6f26 (HEAD -> master)
Author: Skoneek3r <david.dedeus@free.fr>
Date:   Fri Feb 4 15:46:07 2022 +0100 ajout des vagrantfiles}

\section{TP 2}
le \textbf{working directory} fonctionne bien mais il n'a pas changé.

la branche crée existe toujours mais on peut la supprimer à l'aide de la commande 
\textbf{git branch -d nom-de-la-branche}

\section{TP 3}
il y a un conflit entre les branches.
\subsection{log final}
\textit{commit d4c33bd6503f98ec1ba2a28adad6428b2ecd660b (HEAD -> forward-new-port)
Author: Skoneek3r <david.dedeus@free.fr>
Date:   Fri Feb 4 16:39:56 2022 +0100

            modifié :         outils_libres_david_dedeus.tex
            modifié :         vagrant/Vagrantfile

commit 39b9d8d20057f05c82f4fe274321c3c8d3c1ccd3
Author: Skoneek3r <david.dedeus@free.fr>
Date:   Fri Feb 4 16:34:45 2022 +0100

    8081 pour srv

commit 0b7a33a2e6f7c228dceb08d0eb079476ee5a6f26 (vagrant, master)
Author: Skoneek3r <david.dedeus@free.fr>
Date:   Fri Feb 4 15:46:07 2022 +0100

    ajout des vagrantfiles}


\end{document}
